\chapter{Algoritmos}

\section{Desigualdad del Triángulo}

\section{Delaunay}

\section{Línea de Barrido}

\subsection{Conjunto de Intersecciones}

\subsection{El problema de los $n-\text{círculos}$ }

El problema de los $n-\text{círculos}$ dice lo siguiente:
\begin{quote} 
Dado un conjunto $C$ de $n$ círculos y un conjunto $P$ de $m$ puntos, 
encontrar todos los círculos en $C$ que están vacíos.
\end{quote}
El algoritmo explicado en esta sección está tomado del artículo de 2001 de Rao y Mukhopadhyay \citep{Rao2001}.

\subsubsection{Datos del ejemplo}

Para definir a los $n$ círculos, vamos a usar la notación $c:((h,k),r)$, donde las coordenadas de su centro son $(h,k)$, y su radio $r$, tal que 

\begin{equation}
    c_{limit}={(x, y)\in \mathbb{R}^2:(x-h)^2+(y-k)^2=r^2}
\end{equation}
\begin{equation}
    c_{fill}={(x, y)\in \mathbb{R}^2:(x-h)^2+(y-k)^2<r^2}
\end{equation}
\begin{equation}
    c=c_{limit} \cup c_{fill}
\end{equation}
Para nuestro ejemplo tenemos 16 círculos y 6 puntos de entrada mostrados en la figura \ref{circles_data}, cuyos datos se muestran en la siguiente tabla:

\begin{table}[h]
\centering
\begin{tabular}{|l|l|}
\hline
\textbf{Círculos} & \textbf{Puntos} \\
\hline
$c_1:((9.5, 30.5),3.5)$ & $p_1=(14, 30.5)$ \\
$c_2:((11, 30.5),3.5)$ & $p_2=(9, 17)$ \\
$c_3:((18, 24),5)$      & $p_3=(18, 20)$ \\
$c_4:((4, 23),3)$       & $p_4=(28, 18)$ \\
$c_5:((4, 17),4)$       & $p_5=(11, 5)$ \\
$c_6:((7, 17),4.5)$     &   \\
$c_7:((12.5,17),3)$     &   \\
$c_8:((18, 15),6)$      &   \\
$c_9:((27,19),3)$       &   \\
$c_{10}:((30,17),3)$    &   \\
$c_{11}:((4, 11),3)$    &   \\
$c_{12}:((12, 9),2)$    &   \\
$c_{13}:((16,9),3)$     &   \\
$c_{14}:((6, 4),3)$     &   \\
$c_{15}:((7.5, 4),2.5)$ &   \\
$c_{16}:((18, 5),5)$    &   \\
\hline
\end{tabular}
\caption{Datos de círculos y puntos}
\end{table}

\begin{figure}
    \centering
    \includegraphics[width=1\linewidth]{imagenes/circles_01_data.pdf}
    \caption{Datos del ejemplo del problema de los $n-\text{círculos}$ }
    \label{circles_data}
\end{figure}

\subsubsection{Inicialización}

Para nuestro caso los conjuntos $C$ y $P$ son:

\begin{equation}
    C = \{ c_1, c_2, c_3, c_4, c_5, c_6, c_7, c_8, c_9, c_{10}, c_{11}, c_{12}, c_{13}, c_{14}, c_{15}, c_{16}\}
\end{equation}

\begin{equation}
    P = \{ p_1, p_2, p_3, p_4, p_5, p_6\}
\end{equation}

\begin{definition}
El estado de línea de barrido $\mathcal{L}$ es un ordenamiento del conjunto de regiones $\Theta$ que atraviesa la línea vertical $x=a$. En donde, para dos regiones $r_i\in \Theta$ y $r_j\in \Theta$, se dice que $r_i <_a r_j$ si la intersección de la región $r_i$ con la línea vertical $x=a$ está por debajo de la intersección de la región $r_j$ con esa misma línea.
\end{definition}

\begin{figure}
    \centering
    \includegraphics[width=1\linewidth]{imagenes/circles_02_regions.pdf}
    \caption{Regiones en las que los 16 círculos dividen a $\mathbb{R}^2$ }
    \label{circles_region}
\end{figure}

La figura \ref{circles_region} muestra las 35 zonas en las que los 16 círculos dividen $\mathbb{R}^2$, cabe notar que la región $r_1$ se compone de la siguiente manera:

\begin{equation}
    r_0={(x, y)\in \mathbb{R}^2: (x,y)\notin c_{fill}, c\in C}
\end{equation}

\begin{definition}
Para cada región $r_i \in \mathcal{L}$, llamamos $C_{r_i}$ al conjunto de círculos activos que contienen esa región.
\end{definition}

Por definición, para nuestro ejemplo: $C_{r_1}=\emptyset$.

\begin{definition}
Un punto de evento $q$ puede ser:

\begin{itemize}
    \item un punto de entrada ($p_j$)
    \item un punto de intersección entre dos círculos ($p_{int-a_{i,k}}, p_{int-b_{i,k}}$)
    \item el punto más a la izquierda de un círculo (nos referiremos a ellos como puntos izquierdos, $p_{izq-c_i}$)
    \item el punto más a la derecha de un círculo (nos referiremos a ellos como puntos derechos, $p_{der-c_i}$)
\end{itemize}

Estos eventos se mantienen en una fila de prioridad $Q$ que se inicializa con los puntos izquierdos y con los puntos de entrada.
\end{definition}

Con las definiciones anteriores, podemos inferir que para todo $x<p$, donde p es el primer punto en $Q$, 
\begin{equation}
\mathcal{L}={r_1}    
\end{equation}


Para nuestro ejemplo:
\begin{equation}
\begin{aligned}
    \text{Puntos de entrada}
    &=P=\{ p_1, p_2, p_3, p_4, p_5, p_6\}\\
    \text{Puntos de intersección}
    &=\{ p_{int-a_{1,2}}, p_{int-b_{1,2}}, p_{int-a_{4,5}}, p_{int-b_{4,5}}, p_{int-a_{4,6}}, p_{int-b_{4,6}},\\ 
    &\qquad p_{int-a_{3,8}}, p_{int-b_{3,8}}, p_{int-a_{5,6}}, p_{int-b_{5,6}}, p_{int-a_{5,11}}, p_{int-b_{5,11}},\\ 
    &\qquad p_{int-a_{6,7}}, p_{int-b_{6,7}}, p_{int-a_{7,8}}, p_{int-b_{7,8}}, p_{int-a_{8,13}}, p_{int-b_{8,13}},\\ 
    &\qquad p_{int-a_{8,16}}, p_{int-b_{8,16}}, p_{int-a_{9,10}}, p_{int-b_{9,10}}, p_{int-a_{12,13}}, p_{int-b_{12,13}},\\ 
    &\qquad p_{int-a_{14,15}}, p_{int-b_{14,15}}, \}\\
    \text{Puntos izquierdos}
    &=\{p_{izq-c_1}, p_{izq-c_2}, p_{izq-c_3}, p_{izq-c_4}, p_{izq-c_5}, p_{izq-c_6}, p_{izq-c_7},\\ 
    &\qquad p_{izq-c_8}, p_{izq-c_9}, p_{izq-c_{10}}, p_{izq-c_{11}}, p_{izq-c_{12}}, p_{izq-c_{13}}, p_{izq-c_{14}},\\ 
    &\qquad p_{izq-c_{15}}, p_{izq-c_{16}} \}\\
    \text{Puntos derechos}
    &=\{p_{der-c_1}, p_{der-c_2}, p_{der-c_3}, p_{der-c_4}, p_{der-c_5}, p_{der-c_6}, p_{der-c_7},\\ 
    &\qquad p_{der-c_8}, p_{der-c_9}, p_{der-c_{10}}, p_{der-c_{11}}, p_{der-c_{12}}, p_{der-c_{13}}, p_{der-c_{14}},\\ 
    &\qquad p_{der-c_{15}}, p_{der-c_{16}} \}\\
\end{aligned}
\end{equation}

La figura \ref{circles_data} muestra los puntos de entrada; la figura \ref{circles_izqder} muestra los puntos izquierdos y derechos, y la figura \ref{circles_inter} muestra los puntos de intersección. Usando esta notación podemos construir $Q$

\begin{figure}
    \centering
    \includegraphics[width=1\linewidth]{imagenes/circles_03_izq_der.pdf}
    \caption{Puntos izquierdos y derechos }
    \label{circles_izqder}
\end{figure}

\begin{figure}
    \centering
    \includegraphics[width=1\linewidth]{imagenes/circles_04_int.pdf}
    \caption{Puntos de intersección}
    \label{circles_inter}
\end{figure}

\begin{equation}
\begin{aligned}
Q
  &= (\text{Puntos izquierdos}\cup\text{Puntos de entrada})_{ordenados}\\
  &= \bigl(\{ p_{izq-c_1}, p_{izq-c_2}, p_{izq-c_3}, p_{izq-c_4}, p_{izq-c_5}, p_{izq-c_6}, p_{izq-c_7}, p_{izq-c_8}, p_{izq-c_9}, p_{izq-c_{10}}, \\
  &\qquad  p_{izq-c_{11}}, p_{izq-c_{12}}, p_{izq-c_{13}}, p_{izq-c_{14}}, p_{izq-c_{15}}, p_{izq-c_{16}}\} \cup \{ p_1, p_2, p_3, p_4, p_5, p_6 \}\bigr)_{\text{ordenados}} \\
  &= \bigl(\{ p_{izq-c_1}, p_{izq-c_2}, p_{izq-c_3}, p_{izq-c_4}, p_{izq-c_5}, p_{izq-c_6}, p_{izq-c_7}, p_{izq-c_8}, p_{izq-c_9}, p_{izq-c_{10}}, \\
  &\qquad  p_{izq-c_{11}}, p_{izq-c_{12}}, p_{izq-c_{13}}, p_{izq-c_{14}}, p_{izq-c_{15}}, p_{izq-c_{16}}, p_1, p_2, p_3, p_4, p_5, p_6 \}\bigr)_{\text{ordenados}} \\
  &= \{ p_{izq-c_{5}}, p_{izq-c_{11}}, p_{izq-c_{4}}, p_{izq-c_{6}}, p_{izq-c_{14}}, p_{izq-c_{15}}, p_{izq-c_{1}}, p_{izq-c_{2}}, p_{2}, p_{izq-c_{7}}, \\
  &\qquad p_{izq-c_{12}}, p_{6}, p_{izq-c_{8}}, p_{izq-c_{16}}, p_{izq-c_{13}}, p_{izq-c_{3}}, p_1, p_5, p_3, p_{izq-c_{9}}, p_{izq-c_{10}}, p_4  \}
\end{aligned}
\end{equation}


\begin{definition}
Se dice que un círculo $c$ está activo si se encuentra vacío, o si aún no se ha detectado ningún testigo, esto es, un punto de entrada dentro del círculo.

Al conjunto de círculos activos lo denotamos por $A_c$. 
\end{definition}

Al inicio $A_c=C$, conforme evaluemos a los puntos de entrada éste conjunto se irá ``podando".

\subsubsection{Desarrollo}
Antes de iniciar, nuestros valores son los siguientes:

\begin{equation}
\begin{aligned}
Q 
    &= \{ (0, 17, \text{izq}, c_5), (1, 11, \text{izq}, c_{11}), (1, 23, \text{izq}, c_{4}), (2.5, 17, \text{izq}, c_{6}), (3, 4, \text{izq}, c_{14}),\\ 
    &\qquad (5, 4, \text{izq}, c_{15}), (6, 30.5, \text{izq}, c_{1}), (7.5, 30.5, \text{izq}, c_{2}), (9, 17, \text{ent}, p_{2}), (9.5, 17, \text{izq}, c_{7}),\\
    &\qquad (10, 9, \text{izq}, c_{12}), (11, 5, \text{ent}, p_{6}), (12, 15, \text{izq}, c_{8}), (13, 5, \text{izq}, c_{16}), (13, 9, \text{izq}, c_{13}),\\
    &\qquad (13, 24, \text{izq}, c_{3}), (14, 30.5, \text{ent}, p_{1}), (18, 9.5, \text{ent}, p_{5}), (18, 20, \text{ent}, p_{3}), (24, 19, \text{izq}, c_{9}),\\
    &\qquad (27, 17, \text{izq}, c_{10}), (28,18, \text{ent}, p_{4}) \}\\
A_c
    &=C=\{c_1, c_2, c_3, c_4, c_5, c_6, c_7, c_8, c_9, c_{10}, c_{11}, c_{12}, c_{13}, c_{14}, c_{15}, c_{16}\}\\
\mathcal{L}
    &=\{r_0\}
\end{aligned}
\end{equation}

\newpage
\paragraph{Paso 1: $\mathbf{x=0}$, Algoritmo de punto izquierdo}

{\scriptsize
\begin{enumerate}
    \item $(0, 17, \text{izq}, c_{5})$
    \begin{enumerate}[itemsep=-2.5pt, topsep=-2.5pt]
    
        \item Sea $e \in Q$ el punto izquierdo del círculo $c_{5}$
        \begin{itemize}
            \item $e=(0, 17, \text{izq}, c_{5})$
        \end{itemize}
        \item Incertar en Q el punto derecho del círculo $c_5$
        \begin{itemize}
            \item $Q=Q\cup\{(8, 17, \text{der}, c_5)\}$
        \end{itemize}
        \item $r =$ la región en $\mathcal{L}$ que contiene a $e$
        \begin{itemize}
            \item $\mathcal{L}=\{r_0\}\Rightarrow r=r_0$ 
        \end{itemize}
        \item $r_{up}=$ la región arriba de $c$ de $r$
        \begin{itemize}
            \item $r_{up}=r_0$
        \end{itemize}
        \item $r_{middle}=$ la región en $c$ de $r$
        \begin{itemize}
            \item $r_{middle}=r_{c5}$
        \end{itemize}
        \item $r_{down}=$ la región debajo de $c$ de $r$
        \begin{itemize}
            \item $r_{down}=r_0$
        \end{itemize}
        \item Quitar $r=r_0$ de $\mathcal{L}$.
        \begin{itemize}
            \item $\mathcal{L}=\emptyset$ 
        \end{itemize}
        \item Insertar $r_{up}$, $r_{middle}$, $r_{down}$ en $\mathcal{L}$
        \begin{itemize}
            \item $\mathcal{L}=\mathcal{L}\cup\{r_0, r_{c_5}, r_0\}$
        \end{itemize}
        \item $C_{r_{up}} = C_{r_{down}} = C_r$
        \begin{itemize}
            \item $C_{r_{up}} = C_{r_{down}} = C_r=\emptyset$
        \end{itemize}
        \item $C_{r_{middle}} = C_r \cup \{c_5\}$
        \begin{itemize}
            \item $C_{r_{middle}} = \emptyset \cup \{c_5\}=\{c_5\}$
        \end{itemize}
        \item Si $c_5$ intersecta el arco circular que limita por arriba a $r_{up}$, entonces inserta este punto de intersección en $Q$
        \begin{itemize}
            \item No aplica
        \end{itemize}
        \item Si $c_5$ intersecta el arco circular que limita por abajo a $r_{down}$, entonces inserta este punto de intersección en $Q$
        \begin{itemize}
            \item No aplica
        \end{itemize}
        \item Fin
    \end{enumerate}
\end{enumerate}


\vspace{1.5em} 
Al final del Paso 1 tenemos:

\begin{equation}
\begin{aligned}
Q 
    &= \{ (0, 17, \text{izq}, c_5), (1, 11, \text{izq}, c_{11}), (1, 23, \text{izq}, c_{4}), (2.5, 17, \text{izq}, c_{6}), (3, 4, \text{izq}, c_{14}),\\ 
    &\qquad (5, 4, \text{izq}, c_{15}), (6, 30.5, \text{izq}, c_{1}), (7.5, 30.5, \text{izq}, c_{2}), (8, 17, \text{der}, c_5), (9, 17, \text{ent}, p_{2}), \\
    &\qquad (9.5, 17, \text{izq}, c_{7}), (10, 9, \text{izq}, c_{12}), (11, 5, \text{ent}, p_{6}), (12, 15, \text{izq}, c_{8}), (13, 5, \text{izq}, c_{16}), \\
    &\qquad (13, 9, \text{izq}, c_{13}), (13, 24, \text{izq}, c_{3}), (14, 30.5, \text{ent}, p_{1}), (18, 9.5, \text{ent}, p_{5}), (18, 20, \text{ent}, p_{3}), \\
    &\qquad (24, 19, \text{izq}, c_{9}), (27, 17, \text{izq}, c_{10}), (28,18, \text{ent}, p_{4}) \}\\
A_c
    &=C=\{c_1, c_2, c_3, c_4, c_5, c_6, c_7, c_8, c_9, c_{10}, c_{11}, c_{12}, c_{13}, c_{14}, c_{15}, c_{16}\}\\
\mathcal{L}
    &=\{r_0, r_{c_5}, r_0\}
\end{aligned}
\end{equation}
}

\newpage

\paragraph{Paso 2: $x=1$, Algoritmo de punto izquierdo}

{\scriptsize

\begin{enumerate}
    \item $(1, 11, \text{izq}, c_{11})$
    \begin{enumerate}
        \item Sea $e \in Q$ el punto izquierdo del círculo $c_{11}$
        \begin{itemize}
            \item $e=(1, 11, \text{izq}, c_{11})$
        \end{itemize}
        \item Incertar en Q el punto derecho del círculo $c_{11}$
        \begin{itemize}
            \item $Q=Q\cup\{(7, 11, \text{der}, c_{11})\}$
        \end{itemize}
        \item $r =$ la región en $\mathcal{L}$ que contiene a $e$
        \begin{itemize}
            \item $\mathcal{L}=\{r_{0_1}, r_{c_5}, r_{0_2}\} \Rightarrow r=r_{0_1}$ 
        \end{itemize}
        \item $r_{up}=$ la región arriba de $c$ de $r$
        \begin{itemize}
            \item $r_{up}=r_0$
        \end{itemize}
        \item $r_{middle}=$ la región en $c$ de $r$
        \begin{itemize}
            \item $r_{middle}=r_{c_{11}}$
        \end{itemize}
        \item $r_{down}=$ la región debajo de $c$ de $r$
        \begin{itemize}
            \item $r_{down}=r_0$
        \end{itemize}
        \item Quitar $r=r_{0_1}$ de $\mathcal{L}$.
        \begin{itemize}
            \item $\mathcal{L}=\{r_{c_5}, r_{0_2}\}$ 
        \end{itemize}
        \item Insertar $r_{up}$, $r_{middle}$, $r_{down}$ en $\mathcal{L}$
        \begin{itemize}
            \item $\mathcal{L}=\mathcal{L}\cup\{r_0, r_{c_{11}}, r_0\}=\{r_0, r_{c_{11}}, r_0, r_{c_5}, r_0\}$
        \end{itemize}
        \item $C_{r_{up}} = C_{r_{down}} = C_r$
        \begin{itemize}
            \item $C_{r_{up}} = C_{r_{down}} = C_r=\emptyset$
        \end{itemize}
        \item $C_{r_{middle}} = C_r \cup \{c_{11}\}$
        \begin{itemize}
            \item $C_{r_{middle}} = \emptyset \cup \{c_{11}\}=\{c_{11}\}$
        \end{itemize}
        \item Si $c_{11}$ intersecta el arco circular que limita por arriba a $r_{up}$, entonces inserta este punto de intersección en $Q$
        \begin{itemize}
            \item $c_{11}$ intersecta a $c_5$
            \item $Q=Q\cup\{(2.2224, 13.4167, \text{int}, c_{5y11})\}$
        \end{itemize}
        \item Si $c_{11}$ intersecta el arco circular que limita por abajo a $r_{down}$, entonces inserta este punto de intersección en $Q$
        \begin{itemize}
            \item No aplica
        \end{itemize}
        \item Fin
    \end{enumerate}

Al final tenemos:

\begin{equation}
\begin{aligned}
Q 
    &= \{ (0, 17, \text{izq}, c_5), (1, 11, \text{izq}, c_{11}), (1, 23, \text{izq}, c_{4}), (2.2224, 13.4167, \text{int}, c_{5y11}), (2.5, 17, \text{izq}, c_{6}), \\ 
    &\qquad (3, 4, \text{izq}, c_{14}), (5, 4, \text{izq}, c_{15}), (6, 30.5, \text{izq}, c_{1}), (7, 11, \text{der}, c_11), (7.5, 30.5, \text{izq}, c_{2}), \\
    &\qquad (8, 17, \text{der}, c_5), (9, 17, \text{ent}, p_{2}), (9.5, 17, \text{izq}, c_{7}), (10, 9, \text{izq}, c_{12}), (11, 5, \text{ent}, p_{6}),\\
    &\qquad  (12, 15, \text{izq}, c_{8}), (13, 5, \text{izq}, c_{16}), (13, 9, \text{izq}, c_{13}), (13, 24, \text{izq}, c_{3}), (14, 30.5, \text{ent}, p_{1}), \\
    &\qquad (18, 9.5, \text{ent}, p_{5}), (18, 20, \text{ent}, p_{3}), (24, 19, \text{izq}, c_{9}), (27, 17, \text{izq}, c_{10}), (28,18, \text{ent}, p_{4}) \}\\
A_c
    &=C=\{c_1, c_2, c_3, c_4, c_5, c_6, c_7, c_8, c_9, c_{10}, c_{11}, c_{12}, c_{13}, c_{14}, c_{15}, c_{16}\}\\
\mathcal{L}
    &=\{r_0, r_{c_{11}}, r_0, r_{c_5}, r_0\}
\end{aligned}
\end{equation}


\item $(1, 23, \text{izq}, c_{4})$

\begin{enumerate}
    \item Sea $e \in Q$ el punto izquierdo del círculo $c_{4}$
    \begin{itemize}
        \item $e=(1, 23, \text{izq}, c_{4})$
    \end{itemize}
    \item Incertar en Q el punto derecho del círculo $c_{4}$
    \begin{itemize}
        \item $Q=Q\cup\{(7, 23, \text{der}, c_{4})\}$
    \end{itemize}
    \item $r =$ la región en $\mathcal{L}$ que contiene a $e$
    \begin{itemize}
        \item $\mathcal{L}=\{r_0, r_{c_{11}}, r_0, r_{c_5}, r_{0_e}\} \Rightarrow r=r_{0_e}$ 
    \end{itemize}
    \item $r_{up}=$ la región arriba de $c$ de $r$
    \begin{itemize}
        \item $r_{up}=r_0$
    \end{itemize}
    \item $r_{middle}=$ la región en $c$ de $r$
    \begin{itemize}
        \item $r_{middle}=r_{c_{4}}$
    \end{itemize}
    \item $r_{down}=$ la región debajo de $c$ de $r$
    \begin{itemize}
        \item $r_{down}=r_0$
    \end{itemize}
    \item Quitar $r=r_{0_e}$ de $\mathcal{L}$.
    \begin{itemize}
        \item $\mathcal{L}=\{ r_0, r_{c_{11}}, r_0, r_{c_5}\}$ 
    \end{itemize}
    \item Insertar $r_{up}$, $r_{middle}$, $r_{down}$ en $\mathcal{L}$
    \begin{itemize}
        \item $\mathcal{L}=\mathcal{L}\cup\{r_0, r_{c_{4}}, r_0\}=\{r_0, r_{c_{11}}, r_0, r_{c_5}, r_0, r_{c_{4}}, r_0\}$
    \end{itemize}
    \item $C_{r_{up}} = C_{r_{down}} = C_r$
    \begin{itemize}
        \item $C_{r_{up}} = C_{r_{down}} = C_r=\emptyset$
    \end{itemize}
    \item $C_{r_{middle}} = C_r \cup \{c_{4}\}$
    \begin{itemize}
        \item $C_{r_{middle}} = \emptyset \cup \{c_{4}\}=\{c_{4}\}$
    \end{itemize}
    \item Si $c_{4}$ intersecta el arco circular que limita por arriba a $r_{up}$, entonces inserta este punto de intersección en $Q$
    \begin{itemize}
        \item No aplica
    \end{itemize}
    \item Si $c_{4}$ intersecta el arco circular que limita por abajo a $r_{down}$, entonces inserta este punto de intersección en $Q$
    \begin{itemize}
        \item $c_{11}$ intersecta a $c_5$
        \item $Q=Q\cup\{(2.2224, 20.5833, \text{int}, c_{4y5})\}$
    \end{itemize}
    \item Fin
\end{enumerate}
\end{enumerate}
Al final del Paso 2 tenemos:

\begin{equation}
\begin{aligned}
Q 
    &= \{ (0, 17, \text{izq}, c_5), (1, 11, \text{izq}, c_{11}), (1, 23, \text{izq}, c_{4}), (2.2224, 13.4167, \text{int}, c_{5y11}),  \\ 
    &\qquad (2.2224, 20.5833, \text{int}, c_{4y5}), (2.5, 17, \text{izq}, c_{6}), (3, 4, \text{izq}, c_{14}), (5, 4, \text{izq}, c_{15}), (6, 30.5, \text{izq}, c_{1}),  \\
    &\qquad (7, 11, \text{der}, c_11), (7, 23, \text{der}, c_{4}), (7.5, 30.5, \text{izq}, c_{2}), (8, 17, \text{der}, c_5), (9, 17, \text{ent}, p_{2}), \\
    &\qquad  (9.5, 17, \text{izq}, c_{7}), (10, 9, \text{izq}, c_{12}), (11, 5, \text{ent}, p_{6}), (12, 15, \text{izq}, c_{8}), (13, 5, \text{izq}, c_{16}),   \\
    &\qquad (13, 9, \text{izq}, c_{13}), (13, 24, \text{izq}, c_{3}), (14, 30.5, \text{ent}, p_{1}), (18, 9.5, \text{ent}, p_{5}), (18, 20, \text{ent}, p_{3}),\\
    &\qquad (24, 19, \text{izq}, c_{9}), (27, 17, \text{izq}, c_{10}), (28,18, \text{ent}, p_{4}) \}\\
A_c
    &=C=\{c_1, c_2, c_3, c_4, c_5, c_6, c_7, c_8, c_9, c_{10}, c_{11}, c_{12}, c_{13}, c_{14}, c_{15}, c_{16}\}\\
\mathcal{L}
    &=\{r_0, r_{c_{11}}, r_0, r_{c_5}, r_0, r_{c_{4}}, r_0\}
\end{aligned}
\end{equation}
}


\paragraph{Paso 3: $x=2.2224$, Algoritmo de punto de intersección}

{\scriptsize


\begin{enumerate}
    \item $(2.2224, 13.4167, \text{int}, c_{5y11})$
    \begin{enumerate}
        \item Sea $e$ el punto de intersección entre los círculos $c_{11}$ y $c_{5}.$
        \begin{itemize}
            \item $e=(2.2224, 13.4167, \text{int}, c_{5y11})$
        \end{itemize}
        \item $r_i =$ la región en $\mathcal{L}$ que contiene a $e$ en el límite, para $i=c_5, 0, c_{11}$, en orden de arriba a abajo.
        \item Los arcos que limitan a la región $r_0$ cambian: El arco del círculo $c_{11}$ que limitaba a $r_0$ por abajo ahora lo limita por arriba, y el arco del círculo $c_5$ que limitaba a $r_0$ por arriba, ahora lo limita por abajo.
        \item $C_{r_0}'=C_{r_0}-\{c_{5}, c_{11}\}=\emptyset-\{c_{5}, c_{11}\}=\emptyset$
        \item Si $r_{c_{5}yc_{11}} \in c_5$ then $C_{r_0}' \cup \{c_{5}\}$
        \begin{itemize}
            \item $C_{r_0}''=\emptyset \cup \{c_{5}\}=\{c_{5}\}$
        \end{itemize}
        \item Si $r_{c_{5}yc_{11}} \in c_{11}$ then $C_{r_0}'' \cup \{c_{11}\}$
        \begin{itemize}
            \item $C_{r_0}'''=\{c_{5}\} \cup \{c_{11}\}=\{c_{5}, c_{11} \}=C_{r_{c_{5}yc_{11}}}$
            \item se sustituye $r_0$ en $\mathcal{L}$ por $r_{c_{5}yc_{11}}$
        \end{itemize}
        \item Si los arcos que limitan a la región $r_{c_{11}}$ intersectan, agregar ese punto de intersección a $Q$.
            \begin{itemize}
                \item El punto más a la derecha que cumple con $(x,y) \in c_{11}$ es (7,11), que ya se encuentra dentro de $Q$.
            \end{itemize}
        \item Si los arcos que limitan a la región $r_{c_{5}}$ intersectan, agregar ese punto de intersección a $Q$.
            \begin{itemize}
                \item  El punto más a la derecha que cumple con $(x,y) \in c_{5}$ es (8,17), que ya se encuentra dentro de $Q$.
            \end{itemize}
    \end{enumerate}

Al final tenemos:

\begin{equation}
\begin{aligned}
Q 
    &= \{ (0, 17, \text{izq}, c_5), (1, 11, \text{izq}, c_{11}), (1, 23, \text{izq}, c_{4}), (2.2224, 13.4167, \text{int}, c_{5y11}),  \\ 
    &\qquad (2.2224, 20.5833, \text{int}, c_{4y5}), (2.5, 17, \text{izq}, c_{6}), (3, 4, \text{izq}, c_{14}), (5, 4, \text{izq}, c_{15}),   \\
    &\qquad  (6, 30.5, \text{izq}, c_{1}), (7, 11, \text{der}, c_11), (7, 23, \text{der}, c_{4}), (7.5, 30.5, \text{izq}, c_{2}), (8, 17, \text{der}, c_5), \\
    &\qquad (9, 17, \text{ent}, p_{2}),  (9.5, 17, \text{izq}, c_{7}), (10, 9, \text{izq}, c_{12}), (11, 5, \text{ent}, p_{6}), (12, 15, \text{izq}, c_{8}),   \\
    &\qquad (13, 5, \text{izq}, c_{16}), (13, 9, \text{izq}, c_{13}), (13, 24, \text{izq}, c_{3}), (14, 30.5, \text{ent}, p_{1}), (18, 9.5, \text{ent}, p_{5}), \\
    &\qquad (18, 20, \text{ent}, p_{3}), (24, 19, \text{izq}, c_{9}), (27, 17, \text{izq}, c_{10}), (28,18, \text{ent}, p_{4}) \}\\
A_c
    &=C=\{c_1, c_2, c_3, c_4, c_5, c_6, c_7, c_8, c_9, c_{10}, c_{11}, c_{12}, c_{13}, c_{14}, c_{15}, c_{16}\}\\
\mathcal{L}
    &=\{r_0, r_{c_{11}}, r_{c_{5}yc_{11}}, r_{c_5}, r_0, r_{c_{4}}, r_0\}
\end{aligned}
\end{equation}
    \newpage
    \item $(2.2224, 20.5833, \text{int}, c_{4y5})$
    \begin{enumerate}
        \item Sea $e$ el punto de intersección entre los círculos $c_{4}$ y $c_{5}.$
        \begin{itemize}
            \item $e=(2.2224, 13.4167, \text{int}, c_{5y11})$
        \end{itemize}
        \item $r_i =$ la región en $\mathcal{L}$ que contiene a $e$ en el límite, para $i=c_{4}, 0, c_{5}$, en orden de arriba a abajo.
        \item Los arcos que limitan a la región $r_0$ cambian: El arco del círculo $c_{5}$ que limitaba a $r_0$ por abajo ahora lo limita por arriba, y el arco del círculo $c_4$ que limitaba a $r_0$ por arriba, ahora lo limita por abajo.
        \item $C_{r_0}'=C_{r_0}-\{c_{4}, c_{5}\}=\emptyset-\{c_{4}, c_{5}\}=\emptyset$
        \item Si $r_{c_{4}yc_{5}} \in c_4$ then $C_{r_0}' \cup \{c_{4}\}$
        \begin{itemize}
            \item $C_{r_0}''=\emptyset \cup \{c_{4}\}=\{c_{4}\}$
        \end{itemize}
        \item Si $r_{c_{4}yc_{5}} \in c_5$ then $C_{r_0}'' \cup \{c_{5}\}$
        \begin{itemize}
            \item $C_{r_0}'''=\{c_{4}\} \cup \{c_{5}\}=\{c_{4}, c_{5} \}=C_{r_{c_{4}yc_{5}}}$
            \item se sustituye $r_0$ en $\mathcal{L}$ por $r_{c_{4}yc_{5}}$
        \end{itemize}
        \item Si los arcos que limitan a la región $r_{c_{5}}$ intersectan, agregar ese punto de intersección a $Q$.
            \begin{itemize}
                \item El punto más a la derecha que cumple con $(x,y) \in c_{5}$ es (8,17), que ya se encuentra dentro de $Q$.
            \end{itemize}
        \item Si los arcos que limitan a la región $r_{c_{4}}$ intersectan, agregar ese punto de intersección a $Q$.
            \begin{itemize}
                \item El punto más a la derecha que cumple con $(x,y) \in c_{4}$ es (7,23), que ya se encuentra dentro de $Q$.
            \end{itemize}
    \end{enumerate}

Al final del Paso 3 tenemos:

\begin{equation}
\begin{aligned}
Q 
    &= \{ (0, 17, \text{izq}, c_5), (1, 11, \text{izq}, c_{11}), (1, 23, \text{izq}, c_{4}), (2.2224, 13.4167, \text{int}, c_{5y11}),  \\ 
    &\qquad (2.2224, 20.5833, \text{int}, c_{4y5}), (2.5, 17, \text{izq}, c_{6}), (3, 4, \text{izq}, c_{14}), (5, 4, \text{izq}, c_{15}),   \\
    &\qquad  (6, 30.5, \text{izq}, c_{1}), (7, 11, \text{der}, c_11), (7, 23, \text{der}, c_{4}), (7.5, 30.5, \text{izq}, c_{2}), (8, 17, \text{der}, c_5), \\
    &\qquad (9, 17, \text{ent}, p_{2}),  (9.5, 17, \text{izq}, c_{7}), (10, 9, \text{izq}, c_{12}), (11, 5, \text{ent}, p_{6}), (12, 15, \text{izq}, c_{8}),   \\
    &\qquad (13, 5, \text{izq}, c_{16}), (13, 9, \text{izq}, c_{13}), (13, 24, \text{izq}, c_{3}), (14, 30.5, \text{ent}, p_{1}), (18, 9.5, \text{ent}, p_{5}), \\
    &\qquad (18, 20, \text{ent}, p_{3}), (24, 19, \text{izq}, c_{9}), (27, 17, \text{izq}, c_{10}), (28,18, \text{ent}, p_{4}) \}\\
A_c
    &=C=\{c_1, c_2, c_3, c_4, c_5, c_6, c_7, c_8, c_9, c_{10}, c_{11}, c_{12}, c_{13}, c_{14}, c_{15}, c_{16}\}\\
\mathcal{L}
    &=\{r_0, r_{c_{11}}, r_{c_{5}yc_{11}}, r_{c_5}, r_{c_{4}yc_{5}}, r_{c_{4}}, r_0\}
\end{aligned}
\end{equation}
\end{enumerate}

}
\newpage

\paragraph{Paso 4: $\mathbf{x=2.5}$, Algoritmo de punto izquierdo}

{\scriptsize
\begin{enumerate}
    \item $(2.5, 17, \text{izq}, c_{6})$
    \begin{enumerate}[itemsep=-2.5pt, topsep=-2.5pt]
    
        \item Sea $e \in Q$ el punto izquierdo del círculo $c_{6}$
        \begin{itemize}
            \item $e=(2.5, 17, \text{izq}, c_{6})$
        \end{itemize}
        \item Incertar en Q el punto derecho del círculo $c_6$
        \begin{itemize}
            \item $Q=Q\cup\{(11.5, 17, \text{der}, c_6)\}$
        \end{itemize}
        \item $r =$ la región en $\mathcal{L}$ que contiene a $e$
        \begin{itemize}
            \item $\mathcal{L}=\{r_0, r_{c_{11}}, r_{c_{5}yc_{11}}, r_{c_5}, r_0, r_{c_{4}}, r_0\}\Rightarrow r=r_{c_5}$ 
        \end{itemize}
        \item $r_{up}=$ la región arriba de $c$ de $r$
        \begin{itemize}
            \item $r_{up}=r_{c_5}$
        \end{itemize}
        \item $r_{middle}=$ la región en $c$ de $r$
        \begin{itemize}
            \item $r_{middle}=r_{c_5yc_6}$
        \end{itemize}
        \item $r_{down}=$ la región debajo de $c$ de $r$
        \begin{itemize}
            \item $r_{down}=r_{c_5}$
        \end{itemize}
        \item Quitar $r=r_{c_5}$ de $\mathcal{L}$.
        \begin{itemize}
            \item $\mathcal{L}-r_{c_5}=\{r_0, r_{c_{11}}, r_{c_{5}yc_{11}}, r_0, r_{c_{4}}, r_0\}$ 
        \end{itemize}
        \item Insertar $r_{up}$, $r_{middle}$, $r_{down}$ en $\mathcal{L}$
        \begin{itemize}
            \item $\mathcal{L}=\mathcal{L}\cup\{r_{c_5}, r_{c_5yc_6}, r_{c_5}\}$
        \end{itemize}
        \item $C_{r_{up}} = C_{r_{down}} = C_r$
        \begin{itemize}
            \item $C_{r_{up}} = C_{r_{down}} = C_r=\{c_5\}$
        \end{itemize}
        \item $C_{r_{middle}} = C_r \cup \{c_6\}$
        \begin{itemize}
            \item $C_{r_{middle}} = \{c_5\} \cup \{c_6\}=\{c_5, c_6\}$
        \end{itemize}
        \item Si $c_6$ intersecta el arco circular que limita por arriba a $r_{up}$, entonces inserta este punto de intersección en $Q$
        \begin{itemize}
            \item $c_{6}$ intersecta a $c_4$
            \item $Q=Q\cup\{(3.6630, 20.0190, \text{int}, c_{4y6})\}$
        \end{itemize}
        \item Si $c_6$ intersecta el arco circular que limita por abajo a $r_{down}$, entonces inserta este punto de intersección en $Q$
        \begin{itemize}
            \item $c_{6}$ intersecta a $c_11$
            \item $Q=Q\cup\{(3.6630, 13.9810, \text{int}, c_{6y11})\}$
        \end{itemize}
        \item Fin
    \end{enumerate}
\end{enumerate}


\vspace{1.5em} 
Al final del Paso 4 tenemos:

\begin{equation}
\begin{aligned}
Q 
    &= \{ (0, 17, \text{izq}, c_5), (1, 11, \text{izq}, c_{11}), (1, 23, \text{izq}, c_{4}), (2.2224, 13.4167, \text{int}, c_{5y11}), (2.2224, 20.5833, \text{int}, c_{4y5}), \\ 
    &\qquad  (2.5, 17, \text{izq}, c_{6}), (3, 4, \text{izq}, c_{14}), (3.6630, 13.9810, \text{int}, c_{6y11}), (3.6630, 20.0190, \text{int}, c_{4y6}), (5, 4, \text{izq}, c_{15}),   \\
    &\qquad  (6, 30.5, \text{izq}, c_{1}), (7, 11, \text{der}, c_{11}), (7, 23, \text{der}, c_{4}),  \\
    &\qquad (7.5, 30.5, \text{izq}, c_{2}), (8, 17, \text{der}, c_5), (9, 17, \text{ent}, p_{2}),  (9.5, 17, \text{izq}, c_{7}), (10, 9, \text{izq}, c_{12}), (11, 5, \text{ent}, p_{6}),   \\
    &\qquad (11.5, 17, \text{der}, c_6), (12, 15, \text{izq}, c_{8}), (13, 5, \text{izq}, c_{16}), (13, 9, \text{izq}, c_{13}), (13, 24, \text{izq}, c_{3}), (14, 30.5, \text{ent}, p_{1}), \\
    &\qquad (18, 9.5, \text{ent}, p_{5}), (18, 20, \text{ent}, p_{3}), (24, 19, \text{izq}, c_{9}), (27, 17, \text{izq}, c_{10}), (28,18, \text{ent}, p_{4}) \}\\
A_c
    &=C=\{c_1, c_2, c_3, c_4, c_5, c_6, c_7, c_8, c_9, c_{10}, c_{11}, c_{12}, c_{13}, c_{14}, c_{15}, c_{16}\}\\
\mathcal{L}
    &=\{r_0, r_{c_{11}}, r_{c_{5}yc_{11}}, r_{c_5}, r_{c_5yc_6}, r_{c_5}, r_{c_{4}yc_{5}}, r_{c_{4}}, r_0\}
\end{aligned}
\end{equation}
}
\newpage

\paragraph{Paso 5: $\mathbf{x=3}$, Algoritmo de punto izquierdo}

{\scriptsize
\begin{enumerate}
    \item $(3, 4, \text{izq}, c_{14})$
    \begin{enumerate}[itemsep=-2.5pt, topsep=-2.5pt]
    
        \item Sea $e \in Q$ el punto izquierdo del círculo $c_{14}$
        \begin{itemize}
            \item $e=(3, 4, \text{izq}, c_{14})$
        \end{itemize}
        \item Incertar en Q el punto derecho del círculo $c_14$
        \begin{itemize}
            \item $Q=Q\cup\{(9, 4, \text{der}, c_{14})\}$
        \end{itemize}
        \item $r =$ la región en $\mathcal{L}$ que contiene a $e$
        \begin{itemize}
            \item $\mathcal{L}=\{r_0, r_{c_{11}}, r_{c_{5}yc_{11}}, r_{c_5}, r_{c_5yc_6}, r_{c_5}, r_0, r_{c_{4}}, r_0\}\Rightarrow r=r_{0}$ 
        \end{itemize}
        \item $r_{up}=$ la región arriba de $c$ de $r$
        \begin{itemize}
            \item $r_{up}=r_{0}$
        \end{itemize}
        \item $r_{middle}=$ la región en $c$ de $r$
        \begin{itemize}
            \item $r_{middle}=r_{c_{14}}$
        \end{itemize}
        \item $r_{down}=$ la región debajo de $c$ de $r$
        \begin{itemize}
            \item $r_{down}=r_{0}$
        \end{itemize}
        \item Quitar $r=r_{0}$ de $\mathcal{L}$.
        \begin{itemize}
            \item $\mathcal{L}-r_{0}=\{r_{c_{11}}, r_{c_{5}yc_{11}}, r_{c_5}, r_{c_5yc_6}, r_{c_5}, r_0, r_{c_{4}}, r_0\}$ 
        \end{itemize}
        \item Insertar $r_{up}$, $r_{middle}$, $r_{down}$ en $\mathcal{L}$
        \begin{itemize}
            \item $\mathcal{L}=\mathcal{L}\cup\{r_{0}, r_{c_{14}}, r_{0}\}$
        \end{itemize}
        \item $C_{r_{up}} = C_{r_{down}} = C_r$
        \begin{itemize}
            \item $C_{r_{up}} = C_{r_{down}} = C_r=\{0\}$
        \end{itemize}
        \item $C_{r_{middle}} = C_r \cup \{c_{14}\}$
        \begin{itemize}
            \item $C_{r_{middle}} = \emptyset \cup \{c_{14}\}=\{c_{14}\}$
        \end{itemize}
        \item Si $c_{14}$ intersecta el arco circular que limita por arriba a $r_{up}$, entonces inserta este punto de intersección en $Q$
        \begin{itemize}
            \item No aplica
        \end{itemize}
        \item Si $c_14$ intersecta el arco circular que limita por abajo a $r_{down}$, entonces inserta este punto de intersección en $Q$
        \begin{itemize}
            \item No aplica
        \end{itemize}
        \item Fin
    \end{enumerate}
\end{enumerate}


\vspace{1.5em} 
Al final del Paso 5 tenemos:

\begin{equation}
\begin{aligned}
Q 
    &= \{ (0, 17, \text{izq}, c_5), (1, 11, \text{izq}, c_{11}), (1, 23, \text{izq}, c_{4}), (2.2224, 13.4167, \text{int}, c_{5y11}), (2.2224, 20.5833, \text{int}, c_{4y5}), \\ 
    &\qquad  (2.5, 17, \text{izq}, c_{6}), (3, 4, \text{izq}, c_{14}), (3.6630, 13.9810, \text{int}, c_{6y11}), (3.6630, 20.0190, \text{int}, c_{4y6}), (5, 4, \text{izq}, c_{15}),   \\
    &\qquad (6, 30.5, \text{izq}, c_{1}), (7, 11, \text{der}, c_{11}), (7, 23, \text{der}, c_{4}),  \\
    &\qquad (7.5, 30.5, \text{izq}, c_{2}), (8, 17, \text{der}, c_5), (9, 4, \text{der}, c_{14}), (9, 17, \text{ent}, p_{2}),  (9.5, 17, \text{izq}, c_{7}), (10, 9, \text{izq}, c_{12}), (11, 5, \text{ent}, p_{6}),   \\
    &\qquad (11.5, 17, \text{der}, c_6), (12, 15, \text{izq}, c_{8}), (13, 5, \text{izq}, c_{16}), (13, 9, \text{izq}, c_{13}), (13, 24, \text{izq}, c_{3}), (14, 30.5, \text{ent}, p_{1}), \\
    &\qquad (18, 9.5, \text{ent}, p_{5}), (18, 20, \text{ent}, p_{3}), (24, 19, \text{izq}, c_{9}), (27, 17, \text{izq}, c_{10}), (28,18, \text{ent}, p_{4}) \}\\
A_c
    &=C=\{c_1, c_2, c_3, c_4, c_5, c_6, c_7, c_8, c_9, c_{10}, c_{11}, c_{12}, c_{13}, c_{14}, c_{15}, c_{16}\}\\
\mathcal{L}
    &=\{r_{0}, r_{c_{14}}, r_{0}, r_{c_{11}}, r_{c_{5}yc_{11}}, r_{c_5}, r_{c_5yc_6}, r_{c_5}, r_{c_{4}yc_{5}}, r_{c_{4}}, r_0\}
\end{aligned}
\end{equation}
}
\newpage

\paragraph{Paso 6: $x=3.6630$, Algoritmo de punto de intersección}

{\scriptsize


\begin{enumerate}
    \item $(3.6630, 13.9810, \text{int}, c_{6y11})$
    \begin{enumerate}
        \item Sea $e$ el punto de intersección entre los círculos $c_{6}$ y $c_{11}.$
        \begin{itemize}
            \item $e=(3.6630, 13.9810, \text{int}, c_{6y11})$
        \end{itemize}
        \item $r_i =$ la región en $\mathcal{L}$ que contiene a $e$ en el límite, para $i=c_5yc_6, c_5, c_5yc_{11}$, en orden de arriba a abajo.
        \item Los arcos que limitan a la región $r_{c_5}$ cambian: El arco del círculo $c_{11}$ que limitaba a $r_{c_5}$ por abajo ahora lo limita por arriba, y el arco del círculo $c_6$ que limitaba a $r_0$ por arriba, ahora lo limita por abajo.
        \item $C_{r_{c_5}}'=C_{r_{c_5}}-\{c_{6}, c_{11}\}=\{c_5\}-\{c_{6}, c_{11}\}=\{c_5\}$
        \item Si $r_{c_{5}c_{6}yc_{11}} \in c_6$ then $C_{r_{c_5}}' \cup \{c_{6}\}$
        \begin{itemize}
            \item $C_{r_0}''=\{c_5\}\cup \{c_{6}\}=\{c_{6}\}$
        \end{itemize}
        \item Si $r_{c_{5}c_{6}yc_{11}} \in c_{11}$ then $C_{r_{c_5}}'' \cup \{c_{11}\}$
        \begin{itemize}
            \item $C_{r_0}'''=\{c_{5},c_{6}\} \cup \{c_{11}\}=\{c_{5}, c_{6}, c_{11} \}=C_{r_{c_{5}c_{6}yc_{11}}}$
            \item se sustituye $r_{c_5}$ en $\mathcal{L}$ por $r_{c_{5}c_{6}yc_{11}}$
        \end{itemize}
        \item Si los arcos que limitan a la región $r_{c_{5}yc_{6}}$ intersectan, agregar ese punto de intersección a $Q$.
            \begin{itemize}
                \item El punto más a la derecha que cumple con $(x,y) \in c_{5}\cap c_{6}$ es (8,17), que ya se encuentra dentro de $Q$.
            \end{itemize}
        \item Si los arcos que limitan a la región $r_{c_{5}yc_{11}}$ intersectan, agregar ese punto de intersección a $Q$.
            \begin{itemize}
                \item El punto más a la derecha que cumple con $(x,y) \in c_{5}\cap c_{11}$ y en el límite de la región $c_{6}$ es (4.7917, 13.0791), se agrega $(4.7917, 13.0791, \text{int}, c_{5y6})$ a $Q$.
            \end{itemize}
    \end{enumerate}

Al final tenemos:

\begin{equation}
\begin{aligned}
Q 
    &= \{ (0, 17, \text{izq}, c_5), (1, 11, \text{izq}, c_{11}), (1, 23, \text{izq}, c_{4}), (2.2224, 13.4167, \text{int}, c_{5y11}), (2.2224, 20.5833, \text{int}, c_{4y5}), \\ 
    &\qquad  (2.5, 17, \text{izq}, c_{6}), (3, 4, \text{izq}, c_{14}), (3.6630, 13.9810, \text{int}, c_{6y11}), (3.6630, 20.0190, \text{int}, c_{4y6}), (4.7917, 13.0791, \text{int}, c_{5y6}), (5, 4, \text{izq}, c_{15}),   \\
    &\qquad (6, 30.5, \text{izq}, c_{1}), (7, 11, \text{der}, c_{11}), (7, 23, \text{der}, c_{4}),  \\
    &\qquad (7.5, 30.5, \text{izq}, c_{2}), (8, 17, \text{der}, c_5), (9, 4, \text{der}, c_{14}), (9, 17, \text{ent}, p_{2}),  (9.5, 17, \text{izq}, c_{7}), (10, 9, \text{izq}, c_{12}), (11, 5, \text{ent}, p_{6}),   \\
    &\qquad (11.5, 17, \text{der}, c_6), (12, 15, \text{izq}, c_{8}), (13, 5, \text{izq}, c_{16}), (13, 9, \text{izq}, c_{13}), (13, 24, \text{izq}, c_{3}), (14, 30.5, \text{ent}, p_{1}), \\
    &\qquad (18, 9.5, \text{ent}, p_{5}), (18, 20, \text{ent}, p_{3}), (24, 19, \text{izq}, c_{9}), (27, 17, \text{izq}, c_{10}), (28,18, \text{ent}, p_{4}) \}\\
A_c
    &=C=\{c_1, c_2, c_3, c_4, c_5, c_6, c_7, c_8, c_9, c_{10}, c_{11}, c_{12}, c_{13}, c_{14}, c_{15}, c_{16}\}\\
\mathcal{L}
    &=\{r_{0}, r_{c_{14}}, r_{0}, r_{c_{11}}, r_{c_{5}yc_{11}}, r_{c_{5}c_{6}yc_{11}}, r_{c_5yc_6}, r_{c_5}, r_{c_{4}yc_{5}}, r_{c_{4}}, r_0\}
\end{aligned}
\end{equation}

    \newpage
    \item $(3.6630, 20.0190, \text{int}, c_{4y6})$
    \begin{enumerate}
        \item Sea $e$ el punto de intersección entre los círculos $c_{4}$ y $c_{6}.$
        \begin{itemize}
            \item $e=(3.6630, 20.0190, \text{int}, c_{4y6})$
        \end{itemize}
        \item $r_i =$ la región en $\mathcal{L}$ que contiene a $e$ en el límite, para $i=c_{4}yc_{5}, c_{5}, c_{5}yc_{6}$, en orden de arriba a abajo.
        \item Los arcos que limitan a la región $r_{c_{5}}$ cambian: El arco del círculo $c_{6}$ que limitaba a $r_{c_{5}}$ por abajo ahora lo limita por arriba, y el arco del círculo $c_4$ que limitaba a $r_{c_{5}}$ por arriba, ahora lo limita por abajo.
        \item $C_{r_{c_{5}}}'=C_{r_{c_{5}}}-\{c_{4}, c_{6}\}=\{c_{5}\}-\{c_{4}, c_{6}\}=\{c_{5}\}$
        \item Si $r_{c_{4}c_{5}yc_{6}} \in c_4$ then $C_{r_{c_{5}}}' \cup \{c_{4}\}$
        \begin{itemize}
            \item $C_{r_{c_{5}}}''=\{c_{5}\}\cup \{c_{4}\}=\{c_{4}, c_{5}\}$
        \end{itemize}
        \item Si $r_{c_{4}c_{5}yc_{6}} \in c_6$ then $C_{r_{c_{5}}}'' \cup \{c_{6}\}$
        \begin{itemize}
            \item $C_{r_{c_{5}}}'''=\{c_{4}, c_{5}\} \cup \{c_{6}\}=\{c_{4}, c_{5}, c_{6} \}=C_{r_{c_{4}c_{5}yc_{6}}}$
            \item se sustituye $r_{c_{5}}$ en $\mathcal{L}$ por $r_{c_{4}c_{5}yc_{6}}$
        \end{itemize}
        \item Si los arcos que limitan a la región $r_{c_{4}yc_{5}}$ intersectan, agregar ese punto de intersección a $Q$.
            \begin{itemize}
                \item El punto más a la derecha que cumple con $(x,y) \in c_{4}\cap c_{5}$ y en el límite de la región $c_{6}$ es (4.7917, 20.9209), se agrega $(4.7917, 20.9209, \text{int}, c_{5y6})$ a $Q$.
            \end{itemize}
        \item Si los arcos que limitan a la región $r_{c_{5}yc_{6}}$ intersectan, agregar ese punto de intersección a $Q$.
            \begin{itemize}
                \item El punto más a la derecha que cumple con $(x,y) \in c_{5}\cap c_{6}$ es (8,17), que ya se encuentra dentro de $Q$.
            \end{itemize}
    \end{enumerate}

Al final del Paso 6 tenemos:

\begin{equation}
\begin{aligned}
Q 
    &= \{ (0, 17, \text{izq}, c_5), (1, 11, \text{izq}, c_{11}), (1, 23, \text{izq}, c_{4}), (2.2224, 13.4167, \text{int}, c_{5y11}), (2.2224, 20.5833, \text{int}, c_{4y5}), \\ 
    &\qquad  (2.5, 17, \text{izq}, c_{6}), (3, 4, \text{izq}, c_{14}), (3.6630, 13.9810, \text{int}, c_{6y11}), (3.6630, 20.0190, \text{int}, c_{4y6}), (4.7917, 13.0791, \text{int}, c_{5y6}), (4.7917, 20.9209, \text{int}, c_{5y6}), (5, 4, \text{izq}, c_{15}),   \\
    &\qquad (6, 30.5, \text{izq}, c_{1}), (7, 11, \text{der}, c_{11}), (7, 23, \text{der}, c_{4}),  \\
    &\qquad (7.5, 30.5, \text{izq}, c_{2}), (8, 17, \text{der}, c_5), (9, 4, \text{der}, c_{14}), (9, 17, \text{ent}, p_{2}),  (9.5, 17, \text{izq}, c_{7}), (10, 9, \text{izq}, c_{12}), (11, 5, \text{ent}, p_{6}),   \\
    &\qquad (11.5, 17, \text{der}, c_6), (12, 15, \text{izq}, c_{8}), (13, 5, \text{izq}, c_{16}), (13, 9, \text{izq}, c_{13}), (13, 24, \text{izq}, c_{3}), (14, 30.5, \text{ent}, p_{1}), \\
    &\qquad (18, 9.5, \text{ent}, p_{5}), (18, 20, \text{ent}, p_{3}), (24, 19, \text{izq}, c_{9}), (27, 17, \text{izq}, c_{10}), (28,18, \text{ent}, p_{4}) \}\\
A_c
    &=C=\{c_1, c_2, c_3, c_4, c_5, c_6, c_7, c_8, c_9, c_{10}, c_{11}, c_{12}, c_{13}, c_{14}, c_{15}, c_{16}\}\\
\mathcal{L}
    &=\{r_{0}, r_{c_{14}}, r_{0}, r_{c_{11}}, r_{c_{5}yc_{11}}, r_{c_{5}c_{6}yc_{11}}, r_{c_5yc_6}, r_{c_{4}c_{5}yc_{6}}, r_{c_{4}yc_{5}}, r_{c_{4}}, r_0\}
\end{aligned}
\end{equation}
\end{enumerate}

}
\newpage

\paragraph{Paso 7: $x=4.7917$, Algoritmo de punto de intersección}

{\scriptsize


\begin{enumerate}
    \item $(4.7917, 13.0791, \text{int}, c_{5y6})$
    \begin{enumerate}
        \item Sea $e$ el punto de intersección entre los círculos $c_{5}$ y $c_{6}.$
        \begin{itemize}
            \item $e=(4.7917, 13.0791, \text{int}, c_{5y6})$
        \end{itemize}
        \item $r_i =$ la región en $\mathcal{L}$ que contiene a $e$ en el límite, para $i=c_5c_6yc_{11}, c_5yc_{11}, c_{11}$, en orden de arriba a abajo.
        \item Los arcos que limitan a la región $r_{c_5yc_{11}}$ cambian: El arco del círculo $c_{5}$ que limitaba a $r_{c_5yc_{11}}$ por abajo ahora lo limita por arriba, y el arco del círculo $c_6$ que limitaba a $r_{c_5yc_{11}}$ por arriba, ahora lo limita por abajo.
        \item $C_{r_{c_5yc_{11}}}'=C_{r_{c_5yc_{11}}}-\{c_{5}, c_{6}\}=\{c_5, c_{11}\}-\{c_{5}, c_{6}\}=\{c_{11}\}$
        \item Si $r_{c_{4}yc_{6}} \in c_5$ then $C_{r_{c_5yc_{11}}}' \cup \{c_{5}\}$
        \begin{itemize}
            \item $C_{r_{c_5yc_{11}}}''=\{c_11\}$
        \end{itemize}
        \item Si $r_{c_{4}yc_{6}} \in c_{6}$ then $C_{r_{c_5yc_{11}}}'' \cup \{c_{6}\}$
        \begin{itemize}
            \item $C_{r_{c_5yc_{11}}}'''=\{c_{11}\} \cup \{c_{6}\}=\{c_{11}, c_{6} \}=C_{r_{c_{11}yc_{6}}}$
            \item se sustituye $r_{c_5yc_{11}}$ en $\mathcal{L}$ por $r_{c_{11}yc_{6}}$
        \end{itemize}
        \item Si los arcos que limitan a la región $r_{c_{5}c_{6}yc_{11}}$ intersectan, agregar ese punto de intersección a $Q$.
            \begin{itemize}
                \item El punto más a la derecha que cumple con $(x,y) \in c_{5}\cap c_6 \cap c_{11}$ es (5.7776, 13.4167), se agrega $(5.7776, 13.4167, \text{int}, c_{5y11})$ a $Q$.
            \end{itemize}
        \item Si los arcos que limitan a la región $r_{c_{11}}$ intersectan, agregar ese punto de intersección a $Q$.
            \begin{itemize}
                \item El punto más a la derecha que cumple con $(x,y) \in c_{11}$ es (7,11), que ya se encuentra dentro de $Q$.
            \end{itemize}
    \end{enumerate}

Al final tenemos:

\begin{equation}
\begin{aligned}
Q 
    &= \{ (0, 17, \text{izq}, c_5), (1, 11, \text{izq}, c_{11}), (1, 23, \text{izq}, c_{4}), (2.2224, 13.4167, \text{int}, c_{5y11}), (2.2224, 20.5833, \text{int}, c_{4y5}), \\ 
    &\qquad  (2.5, 17, \text{izq}, c_{6}), (3, 4, \text{izq}, c_{14}), (3.6630, 13.9810, \text{int}, c_{6y11}), (3.6630, 20.0190, \text{int}, c_{4y6}), (4.7917, 13.0791, \text{int}, c_{5y6}), (4.7917, 20.9209, \text{int}, c_{5y6}), (5, 4, \text{izq}, c_{15}),   \\
    &\qquad (5.7776, 13.4167, \text{int}, c_{5y11}), (6, 30.5, \text{izq}, c_{1}), (7, 11, \text{der}, c_{11}), (7, 23, \text{der}, c_{4}),  \\
    &\qquad (7.5, 30.5, \text{izq}, c_{2}), (8, 17, \text{der}, c_5), (9, 4, \text{der}, c_{14}), (9, 17, \text{ent}, p_{2}),  (9.5, 17, \text{izq}, c_{7}), (10, 9, \text{izq}, c_{12}), (11, 5, \text{ent}, p_{6}),   \\
    &\qquad (11.5, 17, \text{der}, c_6), (12, 15, \text{izq}, c_{8}), (13, 5, \text{izq}, c_{16}), (13, 9, \text{izq}, c_{13}), (13, 24, \text{izq}, c_{3}), (14, 30.5, \text{ent}, p_{1}), \\
    &\qquad (18, 9.5, \text{ent}, p_{5}), (18, 20, \text{ent}, p_{3}), (24, 19, \text{izq}, c_{9}), (27, 17, \text{izq}, c_{10}), (28,18, \text{ent}, p_{4}) \}\\
A_c
    &=C=\{c_1, c_2, c_3, c_4, c_5, c_6, c_7, c_8, c_9, c_{10}, c_{11}, c_{12}, c_{13}, c_{14}, c_{15}, c_{16}\}\\
\mathcal{L}
    &=\{r_{0}, r_{c_{14}}, r_{0}, r_{c_{11}}, r_{c_{6}yc_{11}}, r_{c_{5}c_{6}yc_{11}}, r_{c_5yc_6}, r_{c_5}, r_{c_{4}yc_{5}}, r_{c_{4}}, r_0\}
\end{aligned}
\end{equation}

    \newpage
    \item $(4.7917, 20.9209, \text{int}, c_{5y6})$
    \begin{enumerate}
        \item Sea $e$ el punto de intersección entre los círculos $c_{5}$ y $c_{6}.$
        \begin{itemize}
            \item $e=(4.7917, 20.9209, \text{int}, c_{5y6})$
        \end{itemize}
        \item $r_i =$ la región en $\mathcal{L}$ que contiene a $e$ en el límite, para $i=c_{4}, c_4yc_{5}, c_{4}c_{5}yc_{6}$, en orden de arriba a abajo.
        \item Los arcos que limitan a la región $r_{c_4yc_{5}}$ cambian: El arco del círculo $c_{6}$ que limitaba a $r_{c_4yc_{5}}$ por abajo ahora lo limita por arriba, y el arco del círculo $c_5$ que limitaba a $r_{c_4yc_{5}}$ por arriba, ahora lo limita por abajo.
        \item $C_{r_{c_4yc_{5}}}'=C_{r_{c_4yc_{5}}}-\{c_{5}, c_{6}\}=\{c_{4}, c_{5}\}-\{c_{5}, c_{6}\}=\{c_{4}\}$
        \item Si $r_{c_{4}yc_{6}} \in c_4$ then $C_{r_{c_{5}}}' \cup \{c_{4}\}$
        \begin{itemize}
            \item $C_{r_{c_4yc_{5}}}''=\{c_{4}\}\cup \{c_{4}\}=\{c_{4}\}$
        \end{itemize}
        \item Si $r_{c_{4}yc_{6}} \in c_6$ then $C_{r_{c_{5}}}'' \cup \{c_{6}\}$
        \begin{itemize}
            \item $C_{r_{c_4yc_{5}}}'''=\{c_{4}\} \cup \{c_{6}\}=\{c_{4}, c_{6} \}=C_{r_{c_{4}yc_{6}}}$
            \item se sustituye $r_{c_{4}yc{5}}$ en $\mathcal{L}$ por $r_{c_{4}yc_{6}}$
        \end{itemize}
        \item Si los arcos que limitan a la región $r_{c_{4}}$ intersectan, agregar ese punto de intersección a $Q$.
            \begin{itemize}
                \item El punto más a la derecha que cumple con $(x,y) \in c_{4}\cap c_{6}$ es (7,23), que ya se encuentra dentro de $Q$.
            \end{itemize}
        \item Si los arcos que limitan a la región $r_{c_{4}c_{5}yc_{6}}$ intersectan, agregar ese punto de intersección a $Q$.
            \begin{itemize}
                \item El punto más a la derecha que cumple con $(x,y) \in c_{4}\cap c_{5} \cap c_{6}$ es (5.7776, 20.5833), se agrega $(5.7776, 20.5833, \text{int}, c_{4y5})$ a $Q$.
            \end{itemize}
    \end{enumerate}

Al final del Paso 7 tenemos:

\begin{equation}
\begin{aligned}
Q 
    &= \{ (0, 17, \text{izq}, c_5), (1, 11, \text{izq}, c_{11}), (1, 23, \text{izq}, c_{4}), (2.2224, 13.4167, \text{int}, c_{5y11}), (2.2224, 20.5833, \text{int}, c_{4y5}), \\ 
    &\qquad  (2.5, 17, \text{izq}, c_{6}), (3, 4, \text{izq}, c_{14}), (3.6630, 13.9810, \text{int}, c_{6y11}), (3.6630, 20.0190, \text{int}, c_{4y6}), (4.7917, 13.0791, \text{int}, c_{5y6}), (4.7917, 20.9209, \text{int}, c_{5y6}), (5, 4, \text{izq}, c_{15}),   \\
    &\qquad (5.7776, 13.4167, \text{int}, c_{5y11}),(5.7776, 20.5833, \text{int}, c_{4y5}), (6, 30.5, \text{izq}, c_{1}), (7, 11, \text{der}, c_{11}), (7, 23, \text{der}, c_{4}),  \\
    &\qquad (7.5, 30.5, \text{izq}, c_{2}), (8, 17, \text{der}, c_5), (9, 4, \text{der}, c_{14}), (9, 17, \text{ent}, p_{2}),  (9.5, 17, \text{izq}, c_{7}), (10, 9, \text{izq}, c_{12}), (11, 5, \text{ent}, p_{6}),   \\
    &\qquad (11.5, 17, \text{der}, c_6), (12, 15, \text{izq}, c_{8}), (13, 5, \text{izq}, c_{16}), (13, 9, \text{izq}, c_{13}), (13, 24, \text{izq}, c_{3}), (14, 30.5, \text{ent}, p_{1}), \\
    &\qquad (18, 9.5, \text{ent}, p_{5}), (18, 20, \text{ent}, p_{3}), (24, 19, \text{izq}, c_{9}), (27, 17, \text{izq}, c_{10}), (28,18, \text{ent}, p_{4}) \}\\
A_c
    &=C=\{c_1, c_2, c_3, c_4, c_5, c_6, c_7, c_8, c_9, c_{10}, c_{11}, c_{12}, c_{13}, c_{14}, c_{15}, c_{16}\}\\
\mathcal{L}
    &=\{r_{0}, r_{c_{14}}, r_{0}, r_{c_{11}}, r_{c_{6}yc_{11}}, r_{c_{5}c_{6}yc_{11}}, r_{c_5yc_6}, r_{c_5}, r_{c_{4}yc_{6}}, r_{c_{4}}, r_0\}
\end{aligned}
\end{equation}
\end{enumerate}

}
\newpage

\paragraph{Paso 8: $\mathbf{x=5}$, Algoritmo de punto izquierdo}

{\scriptsize
\begin{enumerate}
    \item $(5, 4, \text{izq}, c_{15})$
    \begin{enumerate}[itemsep=-2.5pt, topsep=-2.5pt]
    
        \item Sea $e \in Q$ el punto izquierdo del círculo $c_{15}$
        \begin{itemize}
            \item $e=(5, 4, \text{izq}, c_{15})$
        \end{itemize}
        \item Incertar en Q el punto derecho del círculo $c_15$
        \begin{itemize}
            \item $Q=Q\cup\{(10, 4, \text{der}, c_{15})\}$
        \end{itemize}
        \item $r =$ la región en $\mathcal{L}$ que contiene a $e$
        \begin{itemize}
            \item $\mathcal{L}=\{r_{0}, r_{c_{14}}, r_{0}, r_{c_{11}}, r_{c_{6}yc_{11}}, r_{c_{5}c_{6}yc_{11}}, r_{c_5yc_6}, r_{c_5}, r_{c_{4}yc_{6}}, r_{c_{4}}, r_0\}\Rightarrow r=r_{c_{14}}$ 
        \end{itemize}
        \item $r_{up}=$ la región arriba de $c$ de $r$
        \begin{itemize}
            \item $r_{up}=r_{c_{14}}$
        \end{itemize}
        \item $r_{middle}=$ la región en $c$ de $r$
        \begin{itemize}
            \item $r_{middle}=r_{c_{14}yc_{15}}$
        \end{itemize}
        \item $r_{down}=$ la región debajo de $c$ de $r$
        \begin{itemize}
            \item $r_{down}=r_{c_{14}}$
        \end{itemize}
        \item Quitar $r=r_{c_14}$ de $\mathcal{L}$.
        \begin{itemize}
            \item $\mathcal{L}-r_{c_{14}}=\{r_{0}, r_{0}, r_{c_{11}}, r_{c_{6}yc_{11}}, r_{c_{5}c_{6}yc_{11}}, r_{c_5yc_6}, r_{c_5}, r_{c_{4}yc_{6}}, r_{c_{4}}, r_0\}$ 
        \end{itemize}
        \item Insertar $r_{up}$, $r_{middle}$, $r_{down}$ en $\mathcal{L}$
        \begin{itemize}
            \item $\mathcal{L}=\mathcal{L}\cup\{r_{c_{14}}, r_{c_{14}yc_{15}}, r_{c_{14}}\}$
        \end{itemize}
        \item $C_{r_{up}} = C_{r_{down}} = C_r$
        \begin{itemize}
            \item $C_{r_{up}} = C_{r_{down}} = C_r=\{c_{14}\}$
        \end{itemize}
        \item $C_{r_{middle}} = C_r \cup \{c_{15}\}$
        \begin{itemize}
            \item $C_{r_{middle}} = \{c_{14}\} \cup \{c_{15}\}=\{c_{14}, c_{15}\}$
        \end{itemize}
        \item Si $c_{15}$ intersecta el arco circular que limita por arriba a $r_{up}$, entonces inserta este punto de intersección en $Q$
        \begin{itemize}
            \item $c_{15}$ intersecta a $c_{14}$
            \item $Q=Q\cup \{(7.6667, 6.4944,int,c_{14y15})\}$
        \end{itemize}
        \item Si $c_15$ intersecta el arco circular que limita por abajo a $r_{down}$, entonces inserta este punto de intersección en $Q$
        \begin{itemize}
            \item $c_{15}$ intersecta a $c_{14}$
            \item $Q=Q\cup \{(7.6667, 1.5056,int,c_{14y15})\}$
        \end{itemize}
        \item Fin
    \end{enumerate}
\end{enumerate}


\vspace{1.5em} 
Al final del Paso 8 tenemos:

\begin{equation}
\begin{aligned}
Q 
    &= \{ (0, 17, \text{izq}, c_5), (1, 11, \text{izq}, c_{11}), (1, 23, \text{izq}, c_{4}), (2.2224, 13.4167, \text{int}, c_{5y11}), (2.2224, 20.5833, \text{int}, c_{4y5}), \\ 
    &\qquad  (2.5, 17, \text{izq}, c_{6}), (3, 4, \text{izq}, c_{14}), (3.6630, 13.9810, \text{int}, c_{6y11}), (3.6630, 20.0190, \text{int}, c_{4y6}), (4.7917, 13.0791, \text{int}, c_{5y6}), (4.7917, 20.9209, \text{int}, c_{5y6}), (5, 4, \text{izq}, c_{15}),   \\
    &\qquad (5.7776, 13.4167, \text{int}, c_{5y11}),(5.7776, 20.5833, \text{int}, c_{4y5}), (6, 30.5, \text{izq}, c_{1}), (7, 11, \text{der}, c_{11}), (7, 23, \text{der}, c_{4}),  \\
    &\qquad (7.5, 30.5, \text{izq}, c_{2}), (7.6667, 1.5056,int,c_{14y15}), (7.6667, 6.4944,int,c_{14y15}), (8, 17, \text{der}, c_5), (9, 4, \text{der}, c_{14}), (9, 17, \text{ent}, p_{2}),  (9.5, 17, \text{izq}, c_{7}), (10, 9, \text{izq}, c_{12}), (11, 5, \text{ent}, p_{6}),   \\
    &\qquad (11.5, 17, \text{der}, c_6), (12, 15, \text{izq}, c_{8}), (13, 5, \text{izq}, c_{16}), (13, 9, \text{izq}, c_{13}), (13, 24, \text{izq}, c_{3}), (14, 30.5, \text{ent}, p_{1}), \\
    &\qquad (18, 9.5, \text{ent}, p_{5}), (18, 20, \text{ent}, p_{3}), (24, 19, \text{izq}, c_{9}), (27, 17, \text{izq}, c_{10}), (28,18, \text{ent}, p_{4}) \}\\
A_c
    &=C=\{c_1, c_2, c_3, c_4, c_5, c_6, c_7, c_8, c_9, c_{10}, c_{11}, c_{12}, c_{13}, c_{14}, c_{15}, c_{16}\}\\
\mathcal{L}
    &=\{r_{0}, r_{c_{14}}, r_{c_{14}yc_{15}}, r_{c_{14}}, r_{0}, r_{c_{11}}, r_{c_{6}yc_{11}}, r_{c_{5}c_{6}yc_{11}}, r_{c_5yc_6}, r_{c_5}, r_{c_{4}yc_{6}}, r_{c_{4}}, r_0\}
\end{aligned}
\end{equation}
}
\newpage
















\subsubsection{Código en python}









% \begin{theorem}
% Sea $f$ una función continua en $[a,b]$. Entonces...
% \end{theorem}



% \begin{example}
% El número 7 es primo.
% \end{example}

% \begin{remark}
% Esta nota explica un detalle adicional.
% \end{remark}

% \begin{warning}
% ¡Cuidado! Este resultado solo aplica si $n > 0$.
% \end{warning}





% Esto es un ejemplo de una cita \citep{Stuart2018}. 


% En emacs las lineas de abajo permiten compilar desde un archivo de capitulo
% ver: TeX-master-file-ask
%%% Local Variables:
%%% mode: latex
%%% TeX-master: "../maestria"
%%% End:
